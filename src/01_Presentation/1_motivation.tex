\begin{frame}{Leading thought}
    \begin{itemize}
        \item Advanced numerical methods are everywhere in science
            \begin{itemize}
                \vspace{-0.2em}
                \item Data collection \textcolor{grey5}{(Adaptive experimental instruments)}
                \vspace{-0.2em}
                \item Data analysis
                \vspace{-0.2em}
                \item Visualisation
                \vspace{-0.2em}
                \item Simulation
                \vspace{-0.2em}
                \item \ldots
            \end{itemize}
    \vspace{1.5em}
    \item[$\Rightarrow$] \alert{It's better if you understand how they work} and when they fail
    \vspace{1.5em}
        \item Numerical analysis provides \ldots
            \begin{itemize}
                \item \textbf{Mathematical formulation} of physical problems
                \vspace{-0.2em}
                \item \textbf{Numerical algorithms} to solve them
                \vspace{-0.2em}
                \item Mathematical analysis of \textbf{accuracy} and \textbf{errors}
                \vspace{-0.2em}
                \item Insight to
                    \textbf{efficiency} and \textbf{reliability} of computations
            \end{itemize}
    \end{itemize}
\end{frame}

\begin{frame}{\textcolor{black}{We work on:} Computational materials discovery}
    \begin{itemize}
        \item Nowadays discover materials on a computer
    \end{itemize}
    \begin{center}
    \begin{tikzpicture}[
            % background rectangle/.style={fill=white},
            % show background rectangle
        ]
        \node []   at (0, 2.5) (structure) {\includegraphics[width=2cm]{img/Silicon_crystal.jpg}};

        \node [draw=black,fill=black!5]   at (0, 0) (structure) {\includegraphics[width=1cm]{img/pgob_3443.pdf}};
        \begin{scope}[opacity=0.0]
            \node [draw=white]   at (3, 0) (calc) {\includegraphics[width=1cm]{img/pgob_3443.pdf}};
            \node [opacity=0.0] at (3, 0) {\includegraphics[width=2cm]{img/server.pdf}};
            \node at (6, 0) (prop) {conductivity};
            \node [text width=2.5cm] at (6.5, 0) {Suitable for solar cells?};
            \node at (6.1, -2.0) {\includegraphics[width=2cm]{img/Solar_cell.png}};
        \end{scope}

        %\only<2->{
            \node [draw=black,fill=black!5,opacity=0.8]   at (3, 0) {\includegraphics[width=1cm]{img/pgob_3443.pdf}};
            \node [opacity=0.65] at (3, 0) {\includegraphics[width=2cm]{img/server.pdf}};
            \node []   at (3, 2.0) {\large \textbf{$\displaystyle \min_{\Psi} \langle\Psi, H \Psi\rangle$}};
            \draw[->,thick] ($ (structure.east) + (0.1,0) $) -- ($ (calc.west) - (0.4,0)$);
        %}\only<3->{
            \node [text width=2.5cm] at (6.5, 0) {Suitable for solar cells?};
            \draw[->,thick] ($ (calc.east) + (0.4,0) $) -- ($ (prop.west) + (0.2,0) $);
            \node [] at (6.1, -2.0) {\includegraphics[width=2cm]{img/Solar_cell.png}};
        %}
    \end{tikzpicture}
    \end{center}
    \begin{itemize}
        \smaller
        \item \alert{$\simeq$ 30\% of supercomputer usage}
            $\Rightarrow$ Accurate \& efficient numerics
    \end{itemize}
\end{frame}

\begin{frame}{Topics of this course}
    \begin{itemize}
        \item In this course we won't go HPC, but we cover the basics:
    \begin{itemize}
        \item Solving non-linear equations
        \item Interpolation
        \item Numerical differentiation and integration
        \item Solving linear systems
        \item Solving eigenvalue problems
        \item Solving differential equations
    \end{itemize}
    \end{itemize}
    \vspace{2em}
    \begin{itemize}
        \item You're phone / computer / microscope / spectrometer / \ldots \linebreak
            is using all these techniques \emph{right now}
        \vspace{0.5em}
        \item \textcolor{grey5}{(\ldots and unfortunately does not always get it right)}
    \end{itemize}
\end{frame}
