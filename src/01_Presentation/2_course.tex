\begin{frame}{Course organisation}
    \begin{itemize}
        \item \alert{Lecture}: \textbf{Tuesday, 10:15 - 12:00, CM 1 4}
        \vspace{1em}
        \item \alert{Exercise classes}:
            \begin{itemize}
                \vspace{-0.2em}
                \item \textbf{Wednesday, 10:15 - 12:00, BC07 \& 08}
                \vspace{-0.2em}
                \item Important: You need to \textbf{bring your own laptop}
                    \begin{itemize}
                        \vspace{-0.2em}
                        \item You don't own one $\rightarrow$ Later slide
                    \end{itemize}
            \end{itemize}
        \end{itemize}
        \vspace{0.9em}
        \begin{block}{}
        \begin{itemize}
        \item \alert{Moodle:} \url{https://go.epfl.ch/numerical-analysis}
            \begin{itemize}
                \vspace{0.2em}
                \item[$\Rightarrow$] Single source for all information and material
            \end{itemize}
        \end{itemize}
        \end{block}
        \vspace{0.7em}
        \begin{itemize}
        \item \url{https://teaching.matmat.org/numerical-analysis/}
            \begin{itemize}
                \vspace{-0.2em}
                \item[$\Rightarrow$] Online version of the \alert{lecture notes}
            \end{itemize}
        \vspace{1em}
        \item \alert{ED forum:} \url{https://edstem.org/eu/courses/1977}
            \begin{itemize}
                \vspace{-0.2em}
                \item[$\Rightarrow$] Intended for you to help each other
                \vspace{-0.2em}
                \item TAs will only answer to questions after 2 days
            \end{itemize}
    \end{itemize}
\end{frame}

\begin{frame}{Assistants}
    \begin{columns}
    \begin{column}{1.1\textwidth}
    \begin{itemize}
        \item Head teaching assistant
            \begin{itemize}
                \item \textbf{Bruno Ploumhans}, \texttt{bruno.ploumhans@epfl.ch}, MA B1 457
            \end{itemize}
            \vspace{1.5em}
        \item Teaching assistants
            \begin{itemize}
                    \vspace{0.1em}
                    \item \textbf{Andrea Pintus} \texttt{andrea.pintus@epfl.ch}
                    \vspace{0.1em}
                    \item \textbf{Alessio Siviglia}, \texttt{alessio.siviglia@epfl.ch}
                    \vspace{0.1em}
                    \item \textbf{Divya Suman}, \texttt{divya.suman@epfl.ch}
                    \vspace{0.1em}
                    \item \textbf{Leo Digonzelli}, \texttt{leo.digonzelli@epfl.ch}
                    \vspace{0.1em}
                    \item \textbf{Lorenzo Piersante}, \texttt{lorenzo.piersante@epfl.ch}
                    \vspace{0.1em}
                    \item \textbf{Théodore Decaux}, \texttt{theodore.decaux@epfl.ch}
            \end{itemize}
    \end{itemize}
    \end{column}
    \end{columns}
\end{frame}

\begin{frame}{Exercises}
    \begin{itemize}
        \item Exercises: \textbf{Theoretical}
            and a \textbf{computational} component
            \begin{itemize}
                \vspace{-0.3em}
                \item \textcolor{grey5}{(``numerical experiments'')}
            \end{itemize}
        \vspace{0.5em}
        \item \julia programming language: Computational component
            \begin{itemize}
                \vspace{-0.2em}
                \item You \textbf{need to do these on your laptop}
                \vspace{-0.2em}
                \item Installation instructions on moodle
                \vspace{-0.2em}
                \item Introduction during the first exercise session
                \vspace{-0.2em}
                \item \alert{From week 2} we will assume you have a working setup
                    \begin{itemize}
                        \vspace{-0.3em}
                        \item \textcolor{grey5}{This will be checked before the exercise session}
                    \end{itemize}
            \end{itemize}
            \vspace{0.7em}
        \item Exercises sheets: Available \textbf{Tuesday at 12:00}
            \alert{\href{https://go.epfl.ch/numerical-analysis}{on Moodle}}
            \begin{itemize}
                \vspace{-0.2em}
                \item Sheets are generally \textit{not graded}
                    \textcolor{grey5}{(except one $\rightarrow$ see later slide)}
                \vspace{-0.2em}
                \item Master solution available one week later
                \vspace{-0.2em}
                \item Beginning of exercise session:
                    \alert{15mins} to ask questions on previous sheet
            \end{itemize}
            \vspace{0.7em}
        \item \textbf{The exercises are a preparation for the exam}
    \end{itemize}
\end{frame}

\begin{frame}{A few words about Pluto and Julia}
    \begin{center}
        \Large{What a Julia notebook looks like} \\[3em]
        {\normalsize\smaller[3]
            \hspace{-1.8cm}
            \qrcode[height=0.9cm]
            {https://teaching.matmat.org/numerical-analysis/02_Julia.html}
            \hspace{0.2cm}$\rightarrow$\hspace{0.2cm}
            \url{https://teaching.matmat.org/numerical-analysis/02_Julia.html}}
    \end{center}
\end{frame}

\begin{frame}{Obtaining a laptop}
    \begin{itemize}
        \item A \alert{personal laptop is required} for this course
    \end{itemize}
        \vspace{-0.3em}
        \begin{block}{}
            \begin{center}
            \textbf{If you do not have a laptop yet, make sure to \linebreak
                order one as soon as possible.}
            \end{center}
        \end{block}

        \vspace{0.5em}

        \begin{itemize}
                \item \textbf{Poseidon:} \alert{Discounts and loans} when buying a laptop: \linebreak
            {\smaller[2]
            \url{https://www.epfl.ch/campus/services/en/it-services/discount-and-loan-for-personnal-computers/}
            }
        \vspace{0.3em}
        \item \alert{Financial support}:
        \vspace{-0.3em}
        \begin{itemize}
            \item Website on financing your studies: \linebreak
                {\smaller
                \url{https://www.epfl.ch/education/studies/en/financing-study/}
            }
            \vspace{0.3em}
            \item Need-based scholarships: \linebreak
                {\smaller
                \url{https://www.epfl.ch/education/studies/en/support-scholarships/}}
            \vspace{-0.7em}
            \item Canton of Vaud scholarships:
                {\smaller \url{https://www.vd.ch/formation/aides-financieres-aux-etudes-et-a-la-formation-professionnelle-bourses-ou-prets/informations-principales/}}
        \end{itemize}
    \end{itemize}
\end{frame}


\begin{frame}{Course materials}
    \begin{itemize}
        \item All materials will be published on \alert{Moodle:}
            \begin{center}
            \url{https://go.epfl.ch/numerical-analysis}
            \end{center}
        \vspace{1em}
        \item \textbf{Pluto notebooks} used during the lecture
        \item \textbf{Weekly} exercise sheets \textit{(no submission)}
        \item \textbf{Graded} exercise sheet \textit{(submission mandatory)}
        \vspace{1em}
        \item Additional materials (see \href{https://go.epfl.ch/numerical-analysis}{Moodle}):
            \begin{itemize}
                \item Lecture notes of Prof. Fabio Nobile (English \& French)
                \item Tobin A. Driscoll, Richard J. Braun
                    \textit{Fundamentals of Numerical Computation} (2022).
                \item Giray Ökten \textit{First Semester Numerical analysis} (2023).
                \item MIT's \textit{Introduction to computational thinking} lecture
            \end{itemize}
    \end{itemize}
\end{frame}

\begin{frame}{Course evaluation}
    \begin{itemize}
        \item One graded exercise sheet \alert{(20\%)}
            \begin{itemize}
                \item \alert{One graded sheet}
                    around week 11 ($\approx$ \textbf{6 May})
                \item \alert{Implementation and analysis of algorithms}
                \item Focuses on \textit{computational} aspects of the course
                \item Think of it as a \alert{take home exam}
                \item Individual submissions, interviews and spot checks
            \end{itemize}
        \vspace{2em}
        \item Final exam \alert{(80\%)}
            \begin{itemize}
                \item \alert{Written exam}
                    consisting of \textit{theoretical} problems
                \item Pen and paper questions
            \end{itemize}
    \end{itemize}
\end{frame}

\begin{frame}{Learning objectives}
    \begin{itemize}
        \item We study \alert{numerical techniques}
            to solve mathematical problems\\[1.0em]
        \item \textbf{This is not a programming course}.
            Most exercises can be done by \emph{understanding} and \emph{using} code,
            which we provide
        \vspace{0.5em}
        \item \alert{Main objectives} of the course are:
            \begin{itemize}
                \vspace{-0.2em}
                \item Obtain a basic idea how an algorithm works
                \vspace{-0.2em}
                \item Understand its underlying theory
                \vspace{-0.2em}
                \item Know the advantages and disadvantages of each algorithm
                \vspace{-0.2em}
                \item Know how to employ \julia
                    to solve mathematical problems
                \vspace{-0.2em}
                \item Learn how to analyse numerical results critically
                \vspace{-0.2em}
                \item Learn how to choose the best algorithm for each problem
            \end{itemize}
        \vspace{0.5em}
        \item \alert{Fundamental concepts:}
            \begin{itemize}
                \vspace{-0.2em}
                \item \textbf{Stability} of an algorithm
                \vspace{-0.2em}
                \item \textbf{Accuracy} of an algorithm
                \vspace{-0.2em}
                \item \textbf{Cost} of an algorithm
            \end{itemize}
    \end{itemize}
\end{frame}

\begin{frame}{Questions ?}
    \begin{center}
        \huge{Questions ?}
    \end{center}
\end{frame}
