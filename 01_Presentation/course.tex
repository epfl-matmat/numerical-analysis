\begin{frame}{Course organisation}
    \begin{itemize}
        \item \alert{Lecture}: \textbf{Tuesday, 10:15 - 12:00, CM 1 4}
        \vspace{1.5em}
        \item \alert{Exercise classes}:
            \begin{itemize}
                \item \textbf{Wednesday, 10:15 - 12:00, BC07 \& 08}
                \item Computer room, but feel free to bring your own laptops
            \end{itemize}
        \vspace{1.5em}
        \item \alert{Moodle:} \url{https://go.epfl.ch/numerical-analysis}
            \begin{itemize}
                \item[$\Rightarrow$] Single source for all information and material
            \end{itemize}
    \end{itemize}
    % Splitup of students, which rooms we have, etc.
\end{frame}

\begin{frame}{Assistants}
    \begin{columns}
    \begin{column}{1.1\textwidth}
    \begin{itemize}
        \item Head teaching assistant
            \begin{itemize}
                \item \textbf{Cédric Travelletti}, \texttt{cedric.travelletti@epfl.ch}, MA B1 457
            \end{itemize}
            \vspace{1.5em}
        \item Teaching assistants
            \begin{itemize}
                \vspace{0.1em}
                \item \textbf{Victor Bugnion}, \texttt{victor.bugnion@epfl.ch}
                \vspace{0.1em}
                \item \textbf{Adrien Cadet}, \texttt{adrien.cadet@epfl.ch}
                \vspace{0.1em}
                \item \textbf{Leo Digonzelli}, \texttt{leo.digonzelli@epfl.ch}
                \vspace{0.1em}
                \item \textbf{Aline Janvier}, \texttt{aline.janvier@epfl.ch}
                \vspace{0.1em}
                \item \textbf{Parmenion Mavrikakis}, \texttt{parmenion.mavrikakis@epfl.ch}
                \vspace{0.1em}
                \item \textbf{Divya Suman}, \texttt{divya.suman@epfl.ch}
            \end{itemize}
    \end{itemize}
    \end{column}
    \end{columns}
\end{frame}

\begin{frame}{Exercises}
    \begin{itemize}
        \item Exercises: \textbf{Theoretical}
            and a \textbf{computational} component
            \begin{itemize}
                \vspace{-0.3em}
                \item \textcolor{grey5}{(``numerical experiments'')}
            \end{itemize}
        \vspace{0.5em}
        \item \julia programming language: Computational component
            \begin{itemize}
                \vspace{-0.2em}
                \item Introduction during the first exercise session
                \vspace{-0.2em}
                \item Computer room has \julia preinstalled
                \vspace{-0.2em}
                \item Feel free to install and use it on your own machine !
            \end{itemize}
        \vspace{1em}
        \item Exercises sheets: Available
            \alert{\href{https://go.epfl.ch/numerical-analysis}{on Moodle}}
            on the day before the exercise sessions (\textbf{Tuesday})
            \begin{itemize}
                \item \alert{Graded part} has to be returned \alert{via Moodle}
            by the \textbf{following Monday}
            ($\simeq$ 1 week later)
            \end{itemize}
        \vspace{1em}
        \item \textbf{The exercises are a preparation for the exam}
    \end{itemize}
\end{frame}

\begin{frame}{A few words about Pluto and Julia}
    \begin{center}
        \Large{Show Julia notebook}
    \end{center}
\end{frame}


\begin{frame}{Course materials}
    \begin{itemize}
        \item All materials will be published on \alert{Moodle:}
            \begin{center}
            \url{https://go.epfl.ch/numerical-analysis}
            \end{center}
        \vspace{1em}
        \item \textbf{Pluto notebooks} used during the lecture
        \item \textbf{Weekly} exercise sheets
        \item \textbf{Occasionally} Moodle quizes
        \vspace{1em}
        \item Additional materials (see \href{https://go.epfl.ch/numerical-analysis}{Moodle}):
            \begin{itemize}
                \item Lecture notes of Prof. Fabio Nobile (English \& French)
                \item Tobin A. Driscoll, Richard J. Braun
                    \textit{Fundamentals of Numerical Computation} (2022).
                \item Giray Ökten \textit{First Semester Numerical analysis} (2023).
                \item MIT's \textit{Introduction to computational thinking} lecture
            \end{itemize}
    \end{itemize}
\end{frame}

%   0, 0.5, 1
%  every moodle question          (1 P)
%  every subproblem of a question (1 P)

\begin{frame}{Course evaluation}
    \begin{itemize}
        \item Continuous grading of exercises \alert{(25\%)}
            \begin{itemize}
                \item Each sheet has graded and non-graded components
                \item Usually about one graded exercise per week
                \item Mixture of \textbf{theoretical}, \textbf{computational}
                    \& \textbf{multiple choice}
            \end{itemize}
        \vspace{2em}
        \item Final exam \alert{(75\%)}
            \begin{itemize}
                \item \alert{Written exam} % in computer room
                    consisting of \textit{theoretical} and \textit{computational}
                    problems using \julia
            \end{itemize}
    \end{itemize}
\end{frame}

\begin{frame}{Learning objectives}
    \begin{itemize}
        \item We study \alert{numerical techniques}
            to solve mathematical problems\\[1.0em]
        \item \textbf{This is not a programming course}.
            Most exercises can be done by \emph{understanding} and \emph{using} code,
            which we provide
        \vspace{0.5em}
        \item \alert{Main objectives} of the course are:
            \begin{itemize}
                \vspace{-0.2em}
                \item Obtain a basic idea how an algorithm works
                \vspace{-0.2em}
                \item Understand its underlying theory
                \vspace{-0.2em}
                \item Know the advantages and disadvantages of each algorithm
                \vspace{-0.2em}
                \item Know how to employ \julia
                    to solve mathematical problems
                \vspace{-0.2em}
                \item Learn how to analyse numerical results critically
                \vspace{-0.2em}
                \item Learn how to choose the best algorithm for each problem
            \end{itemize}
        \vspace{0.5em}
        \item \alert{Fundamental concepts:}
            \begin{itemize}
                \vspace{-0.2em}
                \item \textbf{Stability} of an algorithm
                \vspace{-0.2em}
                \item \textbf{Accuracy} of an algorithm
                \vspace{-0.2em}
                \item \textbf{Cost} of an algorithm
            \end{itemize}
    \end{itemize}
\end{frame}

\begin{frame}{Questions ?}
    \begin{center}
        \huge{Questions ?}
    \end{center}
\end{frame}
